\documentclass[10pt]{article}

\usepackage[utf8]{inputenc}
\usepackage[english]{babel}
\usepackage{hyperref}
\usepackage{color}

\usepackage{flashmovie}

\begin{document}

{\Huge flashvideo.sty}\\\\

This package allows direct embedding of flash movies into PDF files. It is
designed for use with pdflatex.

\flashmovie[width=10cm,height=10cm,engine=flv-player,auto=1]{saturn5.mp4}

%\vspace{1cm}

Basically it uses the fact that the Adobe Reader 9 contains an embedded Adobe Flash 
player which can be invoked with the ``rich media annotation'' feature which is described 
in ``Adobe Supplement to the ISO 32000 BaseVersion: 1.7 ExtensionLevel: 3''.

\vspace{0.5cm}

This means that you can only use flashmovie.sty in combination
with Adobe Reader 9 and upwards.
% Otherwise your Adobe Reader may die a sudden painfull death...

\vspace{0.5cm}

\textcolor{red}{
  It is recommended to use the latest available version of the Adobe Reader
  to view PDF files generated with flashmovie.sty.
}

\vspace{0.5cm}

P.S.: This sample video is courtesy of the NASA ( \href{http://heasarc.gsfc.nasa.gov/Videos/historical/saturn5.avi}{saturn5.avi} ).

\newpage

The source code used for the video on the previous page is:

\begin{verbatim}
  \flashmovie[width=10cm,height=10cm,engine=flv-player,auto=1]{saturn5.mp4}
\end{verbatim}

This means that the movie is rendered with the \href{http://flv-player.net}{``flv-player''}
whose developer is neolao. This player is distributed under the
\href{http://www.mozilla.org/MPL/}{MPL version 1.1}.
It is included in this package and is the recommended way to use ``flashmovie.sty''
besides directly embedding ``.swf'' files.

\vspace{1cm}

{\Huge Examples}\\\\

\flashmovie[width=8cm,height=5cm,engine=flv-player,auto=0]{saturn5.mp4}
\begin{verbatim}
  \flashmovie[width=8cm,height=5cm,engine=flv-player,auto=0]{saturn5.mp4}
\end{verbatim}
In this example the video is not started before the user clicks on it.

\vspace{1cm}

\flashmovie[width=8cm,height=5cm,engine=flv-player,auto=0,image=saturn.jpg]{saturn5.mp4}
\begin{verbatim}
  \flashmovie[width=8cm,height=5cm,engine=flv-player,auto=0,image=saturn.jpg]{saturn5.mp4}
\end{verbatim}
Here additionally an image is displayed before the movie starts.

\newpage

Rich media annotations are not restricted to videos.
Here for example is a clock written in action script:

\flashmovie[width=8cm,height=5cm]{clock.swf}
\begin{verbatim}
  \flashmovie[width=8cm,height=5cm]{clock.swf}
\end{verbatim}

\end{document}
